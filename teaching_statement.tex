\documentclass[a4paper,12pt]{article} 
\usepackage{amsmath,amssymb, amsthm, mathrsfs, fancyhdr, ulem, gastex, gensymb, harmony, color, enumitem, bm, hyperref, extarrows, makeidx, rotating, wasysym}
\usepackage[encapsulated]{CJK}
\usepackage{ucs}
\usepackage[utf8x]{inputenc}
%\usepackage[UTF8]{ctex}
% use one of bsmi(trad Chinese), gbsn(simp Chinese), min(Japanese), mj(Korean); see:
% /usr/share/texmf-dist/tex/latex/cjk/texinput/UTF8/*.fd
\usepackage{tikz} 
\usetikzlibrary{arrows,decorations.pathmorphing,backgrounds,fit}  
\usetikzlibrary{circuits}
\newcommand\independent{\protect\mathpalette{\protect\independenT}{\perp}} 
\def\independenT#1#2{\mathrel{\rlap{$#1#2$}\mkern2mu{#1#2}}} 


\pdfpagewidth 8.5in
\pdfpageheight 11in
\setlength\topmargin{0in}
\setlength\headheight{0in}
\setlength\headsep{0in}
\setlength\textheight{9.0in}
\setlength\textwidth{6.5in}
\setlength\oddsidemargin{0in}
\setlength\evensidemargin{0in}
\setlength\parindent{0.0in}
\setlength\parskip{0.25in} 




\begin{document}

When I was an undergraduate student, I found that there was a divide between what was expected for lower level and upper level math and statistics courses. The lower level courses seemed to favor mechanical computations, whereas the upper level courses were more proof-based. This led to disconnect for some of my peers who glided by the lower level courses by memorizing formulae, but had to struggle with upper level courses.

Such an experience has shaped how I see teaching, and I have tried to implement what I see as `responsible teaching' in the five years I have been a TA at Cornell.

{\bf Concepts, not computation}

Starting off as a TA in my first few years, I have created handouts for students which explained {\it why} we do certain computations and make certain statements, and tried to bring in as much pop culture (to be relevant) for the students. For example, when I started off TAing an Introductory Statistics class, I brought in case studies from history (how bad statistics led to poor planning for more hospitals), or even bad articles on science. Having such discussions in class allows the material to take on a life of its own, and engages students. 

In upper level courses such as linear models or statistical computing, I tend to start discussion of problem sets by asking students what they feel the question is asking, and what techniques they could use to solve a problem. 

Usually, students volunteer a few ideas, eg ``this simplifies to a problem in matrix algebra", and I would then keep the discussion going until a sketch of the solution is presented. 

The main motivation for such a discussion is to set the right mindset for the weaker students. I have found that going through the ``right answer" immediately may give such students the impression that ``they could do it themselves if pushed", or ``it seems non-obvious, I'm bad at math/stats since I can't see this". On the other hand, knowing that it takes the entire class to come up with an answer definitely encourages these students.

Furthermore, I try to promote the use of Piazza, where students can post questions anonymously. I've led students to know they can ask questions not just about homework, but about lectures and discussion sections as well. 

The goal is to ensure that even the weaker students understand the material and why it might be relevant, even if they cannot do the computations.

{\bf Collaboration with others}

Where possible, as a TA I have sometimes facilitated the creation of groups of students to work together - not just for homework, but to understand the lecture material. I have always believed that explaining concepts to another person reinforces them, and thus I try to get students to work together where possible.

{\bf Responsible homework setting}

Time (and TAs) is always an issue. While setting ``generic" homework questions usually mean less time grading, I strongly feel that this is shortchanging the student.

I tend to write homework questions which are lengthier, giving a short build up to the scenario I envision. For example, when I was TAing a statistical computing course, I brought in concepts of minwise hashing (in terms of sufficient statistics and finding confidence intervals), and multinomial maximum likelihood estimation (using root finding techniques).

While this necessitated one to two pages of extra writeups, I felt that this would be an acceptable tradeoff for students, letting them know that the concepts learnt are not just limited to textbook examples.

This has led to a small side effect. Once or twice, I have had bright students ask a side question on Piazza related to the homework, which could be potential research questions, and I have encouraged them to go further or do research with the relevant professor during summer.

On the grading side, I have written autograders to automatically grade code based on test cases and output, which lessens homework time. I also write write detailed solution sets and common mistakes for these problems, and encourage students to read them.

For lower level courses, this is essential, since TAs / graders only grade a subset of questions. Students then have the option to read the solution set and potential common mistakes. 

{\bf Evaluation of Teaching}

I tend to spend more time on TA related duties due to the setting of homework questions / coming up with solutions for upper level courses. However, I believe this is worth it, since most of the students I interact with feel they have learnt something after taking the respective courses I TA.

On a more practical note, I also look at course evaluations to see how well I'm doing. Usually, the mid-semester feedback would tend to say the material is difficult, or that I'm a harsh grader due to looking more for understanding of concepts. However, the end of semester feedback becomes more positive when students get used to the style of teaching, and homework / exam evaluation. Thus, I see this as a strong testament to the effectiveness of my teaching philosophy. 



\end{document}




